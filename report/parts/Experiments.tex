\documentclass[../body.tex]{subfiles}
\begin{document}
Итак, на основе разработанных модулей были составлены программные конвейеры двух конфигураций:

\begin{itemize}
	\item Train --- для обучения модели;
	\item Predict --- для использования модели, получения предсказаний.
\end{itemize}

С помощью Train конвейера была обучена модель. Подробности обучения в таблице (\ref{FitParamsTable}) ниже.

\begin{table}[H]
	\centering
	\begin{tabular}{|c|c|}
		\hline
		Параметр & Значение \\
		\hline
		Набор тренировочных данных & Google AudioSet \cite{audioset} \\
		\hline
		Мощность обучающей выборки & 19090 \\
		\hline
		Мощность валидационной выборки & 8181 \\
		\hline
		Число эпох & 30 \\
		\hline
		Опитмизатор & ADAM \cite{adam_optimizer} \\
		\hline
		Функция потерь & Бинарная перекрёстная энтропия \cite{wiki:bin_crossentropy} \\
		\hline
	\end{tabular}
	\caption{Подробности обучения модели}\label{FitParamsTable}
\end{table}

В секции (\ref{MetricsSection}), повсященной полученным результатам,  приведены результаты обучения модели.

\end{document}