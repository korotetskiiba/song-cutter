\documentclass[main.tex]{subfiles}
\begin{document}
	\section{Введение}
	\subsection{Актуальность темы}
	\subfile{parts/Relevance.tex}
	\subsection{Цели и задачи работы}
	\subfile{parts/Purpose.tex}
	\section{Основная часть}
	\subsection{Исходные данные}
	\subfile{parts/initialdata.tex}
	\subsection{Архитектура}
	\subfile{parts/architecture.tex}
	\subsection{Использованные методы}
	\subfile{parts/methods.tex}
	\subsection{Проведенные численные эксперименты}
	\subfile{parts/Experiments.tex}
	\subsection{Полученные результаты}
	\subfile{parts/Metrics.tex}
	\section{Заключение}
	\subfile{parts/conclusion.tex}
	\section{Приложения}
	С кодом программы можно ознакомиться в GitHub репозитории с открытым доступом: \url{https://github.com/korotetskiiba/song-cutter}
	\newpage
	\section{Использованная литература}
	\printbibliography
\end{document}